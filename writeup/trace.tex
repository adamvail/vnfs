\section{Tracing Traffic from the Guest to the Network}
\label{sec:trace}

In order to determine why usermode networking incurs so much more overhead than bridged networking, we traced the path of a packet through both networking stacks.
Both types of networking flow through the same functions when the emulated Realtek driver is invoked to put a packet ``on the wire.``
This code is detailed in appendix ? and we describe the differences between the two systems next.
The differences in the flow of a packet change once the emulated rtl8139 driver attempts to put the packet ``on the wire.``

\subsection{Usermode Networking}
\label{subsec:usermode}

Figure ? shows a flowchart of functions that a packet passes through when a guest VM sends a packet to the Internet while using usermode networking.
The control flow starts with the RealTek driver passing the packet to the \texttt{slirp_input} function.
\texttt{Slirp_input} calls two functions before the control flow continues, \texttt{m_get} which in turn calls \texttt{slirp_insque}.
In \texttt{m_get}, memory is allocated and then passed on to \texttt{slirp_insque} to be added a doubly linked list which acts as a packet buffer.
The chunk of memory is then returned to \texttt{slirp_output} where the packet is then copied into it. % reword
It is this allocating, managing, and copying of memory done for every packet in these three functions that causes the majority of the slowdown in usermode networking.

Once the packet has been copied into the buffer space, it is passed ``up the stack`` to \texttt{ip_input}. 
The function uses the packet's IP headers to sanity check different aspects such as length.
If the packet is malformed then it is dropped and the memory used to store it is freed.

If the packet passes the tests in \texttt{ip_input} it is passed along to \texttt{tcp_input}.
Both the IP and TCP headers are stripped from the packet, leaving just the TCP payload, which is passed to \texttt{tcp_output}.
It is at this point that the packets starts to move back ``down the stack`` to be put out onto the wire.
\texttt{Tcp_ouput} is a direect implementation of the TCP RFC.
It does all the standard operations associated with TCP such as calculating the receive window and updating the TCP control block that Qemu maintains.

With the transport layer complete the packet moves to the \texttt{ip_output} where the packet's IP header fields filled in with the host's information.
At this point the packet is fully formed and ready to put on the wire.
It is handed off to \texttt{if_output} which manages two different packet queues, \texttt{fastq} and \texttt{batchq}.
If a packet has the \texttt{IPTOS_LOWDELAY} flag set, then it is placed on the \texttt{fastq}, which is intended for packets from an interactive application.
Otherwise, the packet is placed on the \texttt{batchq}.
The function also makes calls to \texttt{m_get} and \texttt{slirp_insque} to create a separate linked list.
It isn't readily clear as to why the function has created this extra buffer space.

The packet then moves on to \texttt{if_start} which services both the \textt{fastq} and the \texttt{batchq}.
As the names would indicate, the \texttt{fastq} is serviced first.
This function passes the packets from the queues on and then cleans up memory upon return through the \texttt{m_free} and \texttt{slirp_remque} functions.

When a packet is serviced it is passed to \texttt{if_encap}.
Here the packet is prepared for going out on the wire by copying the packet header information into the appropriate header structs.
Finally the packet is passed along to \texttt{slirp_output}. 
Essentially \texttt{slirp_output} just puts the packet on whatever queue is passed into it along with additional slirp state information.
This is done by simply calling \texttt{qemu_send_packet}.
At this point qemu takes over and the rest of the control path is no longer dictated by the fact that usermode networking is in use.

